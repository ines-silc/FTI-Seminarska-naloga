\documentclass[10pt]{beamer}
\usepackage[slovene]{babel}
\usepackage[utf8]{inputenc}
\usepackage[T1]{fontenc}
\usepackage{lmodern}
\usepackage{mathptmx}
\usepackage{helvet}
\usepackage{courier}
\usepackage{hyperref}
\usepackage{tikz}
\usepackage{enumerate}
\usepackage{eurosym}
\setbeamertemplate{caption}[numbered]

\usetheme{CambridgeUS}

\setbeamercolor*{item}{fg=darkred}
\setbeamercolor*{label}{fg=darkred}
\setbeamercolor{caption name}{fg=darkred}

\begin{document}


\title[Univerzitetne fundacije]{Univerzitetne fundacije}
\author[Eva Deželak in Ines Šilc]{Eva Deželak in Ines Šilc}


\begin{frame}
	\titlepage
\end {frame}


\begin{frame}
\frametitle{Načrt predstavitve}
\begin{itemize}
\item Delovanje fundacij
\item Vrste fundacij
\item Razporejenost fundacij po ZDA
\item Naložbene politike univerzitetnih fundacij
\item Odzivnost na finančne šoke
\item Uporaba sredstev
\item Zaključek
\item Viri
\end{itemize}
\end{frame}

\section{Naložbene politike univerzitetnih fundacij}
\begin{frame}
\frametitle{Naložbene politike univerzitetnih fundacij}
\begin{itemize}
\item Zagotavljajo primerno rast in predvidljivost letnih izplačil
\item Nelikvidne dolgoročne naložbe
\item Ohranjanje prvotne vrednosti oz. \textit{generacijske pravičnost}
\item Stopnja trajnostne porabe sklada mora biti enaka pričakovani skupni donosnosti obdavčenih sredstev, zmanjšani za predvideno stopnjo inflacije
\item Pogosto se za odločijo, da bodo v naslednjem semestru potrošili 5 \% povprečne vrednosti sklada zadnjih treh fiskalnih let.
\item Povečanje potrošnje za toliko, kolikor se je povečala inflacija, povzroči predvidljiv tok dohodkov na operativni proračun.
\item Obdavčitev?
\item Zasebne fundacije se razlikujejo od javnih dobrodelnih organizacij, ki so oproščene davkov, in sicer po ozkih osnovah nadzora in finančne podpore.
\end{itemize}
\end{frame}

\begin{frame}
\begin{itemize}
\item Med letoma 2002 in 2010 se je delež naložbenih sredstev, vloženih v lastniške deleže, zmanjšal s 50 \% na 31 \%.
\item Od leta 2010 se je delež premoženjskih sredstev, vloženih v delnice, povečal na 36 \%.
\item Odstotek sredstev, vloženih v stalni dohodek, se je v obdobju 2002 do 2017 zmanjšal s 23 \% na 8 \%. Medtem ko se je delež sredstev, ki se vlagajo v kapital in stalni dohodek, zmanjšal, se je delež sredstev, vloženih v alternativne strategije, povečeval.
\item Glavni dohodek fundacij so pa seveda donacije.
\item Dostop do strokovnega znanja.
\item Najbolj uspešne fundacije imajo dostop do alternativnih naložb, ki zahtevajo daljša obdobja naložbe in višje minimalne naložbe, kot si jih lahko privošči večina posameznih vlagateljev.
\end{itemize}
\end{frame}

\section{Odzivnost na finančne šoke}
\begin{frame}
\frametitle{Odzivnost na finančne šoke}
\begin{itemize}
\item Po pozitivnih šokih po navadi sledijo njihovim navedenim politikam izplačil.
\item po negativnih šokih številna sredstva aktivno odstopajo od svojih navedenih pravil o izplačilih.
\item Ne najdemo doslednih dokazov, da univerze spreminjajo izplačila darov, da bi nadomestile šoke v druge vire univerzitetnih prihodkov, torej vedenje fundacij ni v skladu s hipotezami glajenja ali samozavarovanja.
\item Prva teorija o fundacijah trdi, da bi morali skrbniki fundacijskih institucije ravnati, kot da je ustanova nesmrtna, in si prizadevati enako obravnavati vse generacije.
\begin{enumerate}
\item Kratkoročni odziv na izplačila na šoke bi moral biti majhen, ker bi morale fundacije s časom širiti dobičke in izgube.
\item Fundacije bi se morale simetrično odzvati na pozitivne in negativne pretrese.
\end{enumerate}
\item Druga teorija univerzitetnih fundacij je, da univerzam omogočajo samozavarovanje, na primer dodelitev sredstev za varovanje pred šokom prihodkov ali uporabo fundacije kot oblike previdnostnih prihrankov, ki jih je mogoče izkoristiti, ko so drugi prihodki nepričakovano nizki.
\end{itemize}
\end{frame}

\section{Viri}
\begin{frame}
\frametitle{Viri}
\begin{itemize}

\item Brown, J. R., Dimmock, S. G., Kang, J. K., \& Weisbenner, S. J. (2014). \textit{How university endowments respond to financial market shocks: Evidence and implications}. American Economic Review, 104(3), 931-62.

\item Cambridge Investment Management Limited. \emph{Cambridge University Endowment Fund, Annual Report 2019}. Pridobljeno 16.~maja~2020 iz \url{https://www.cambridgeinvestmentmanagement.co.uk/files/cuef_annual_report_2019_v2.pdf}.

\item Harvard University. \emph{Endowment}. Pridobljeno 16.~maja~2020 iz \url{https://www.harvard.edu/about-harvard/harvard-glance/endowment}.

\item Harvard University. \emph{Financial report, fiscal year 2019, Harvard University}. Pridobljeno 16.~maja~2020 iz \url{https://www.harvard.edu/sites/default/files/content/fy19_harvard_financial_report.pdf}.
\end{itemize}
\end{frame}

\begin{frame}
\begin{itemize}
\item Investopedia. \emph{How do university endowments work?} Pridobljeno 29.~aprila~2020 iz \url{https://www.investopedia.com/ask/answers/how-do-university-endowments-work/}.

\item Investopedia. \emph{How To Invest Like An Endowment?}, Pridobljeno 13.~maja~2020 iz \url{https://www.investopedia.com/articles/financial-theory/09/ivy-league-endowments-money-management.asp}.

\item Sherlock, M. F., Crandall-Hollick, M. L., Gravelle, J., \& Stupak, J. M. (2015). \textit{College and university endowments: Overview and tax policy options. Washington, DC: Congressional Research Service}.

\item Spitz, W. T. (2010). \emph{Investment policies for college and university endowments. Roles and Responsibilities of the Chief Financial Officer: New Directions for Higher Education}, Number 107, 99, 51.

\item Wikipedia: The Free Encyclopedia. \emph{List of colleges and universities in the United States by endowment},  Pridobljeno 12.~maja~2020 iz \url{https://en.wikipedia.org/wiki/List_of_colleges_and_universities_in_the_United_States_by_endowment}.
\end{itemize}
\end{frame}


\end{document}