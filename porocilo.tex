\documentclass[12pt, a4paper]{article}
\usepackage[utf8]{inputenc}
\usepackage[T1]{fontenc}
\usepackage[slovene]{babel}
\usepackage{times}
\renewcommand{\familydefault}{\rmdefault}
\usepackage{amsmath}
\usepackage{eurosym}
\usepackage{hyperref}
\usepackage{graphicx}
\usepackage[top=2.5cm, bottom=2.5cm, left=3cm, right=2.5cm]{geometry}
\usepackage{indentfirst}
\setlength{\parindent}{0.5cm}


\newcommand{\novukaz}[2]{\underline {#1} \textit{#2}}

\newcounter{stevec}

\newenvironment{novookolje}[2]{\stepcounter{stevec} #1 #2 \thestevec}{}

\begin{document}

\begin{titlepage}
\begin{center}

\large
Univerza v Ljubljani\\
\normalsize
Fakulteta za matematiko in fiziko\\

\vspace{3 cm} 

\large
Eva Deželak in Ines Šilc\\

\vspace{0.5cm}
\LARGE
\textbf{Univerzitetne fundacije}

\vspace{0.5 cm}
\normalsize


\vspace{1.5cm}
\normalsize
SEMINARSKA NALOGA PRI PREDMETU FINANČNI TRGI IN INSTITUCIJE

\vspace{3cm}


\vfill

\large Ljubljana, 2020

\end{center}
\end{titlepage}

\newpage

\tableofcontents

\listoffigures

\newpage

\section{Uvod}

V seminarski nalogi bova raziskali kaj so univerzitetne fundacije, kje se pojavljajo, kakšno je njihovo obnašanje na finančnih trgih, kako so sestavljene njihove naložbene politike in kakšna je dejanska velikost fundacij predvsem v Združenih državah Amerike, kjer so take oblike skladov najizrazitejše.\\

Najstarejše fundacije, ki so aktivne še danes, so ustanovili kralj Henry VIII in njegovi sorodniki. Njegova babica, grofica Richmond, je ustanovila "endowed chairs in divinity"????? na Oxfordu in Cambridgu, Henry VIII pa je na obeh ustanovil "professorships in a variety of disciplines"????. Prvo zabeleženo fundacijo pa je ustanovil Marcus Aurelius za višjo šolo filozofije v Atenah leta 176. \cite{Investopedia}

\section{Kaj so univerzitetne fundacije, njihov namen in delovanje}


Fundacije predstavljajo denar ali druga finančna sredstva, ki se donirajo univerzam. Namenjena so investiranju v rast osnovnega kapitala in zagotavljajo dodatni prihodek za prihodnje investicije in izdatke. Običajno fundacijski skladi pri dodeljevanju sredstev sledijo striktnim pravilom z namenom, da ne bi prevzeli preveč tveganja. \\

Fundacija je denarna ali premoženjska podpora neprofitni organizaciji, ki porabi dobiček od naložb za določen namen. Večina fundacij je zasnovana tako, da glavni znesek ohrani nedotaknjen, medtem ko donos od naložbe porabi za dobrodelne namene. \\

\subsection{Delovanje fundacij}

Večina fundacij ima določene smernice, ki narekujejo, koliko vsakoletnih prihodkov iz naložb lahko porabijo. Na številnih univerzah ta znesek znaša približno 5\% celotne vrednosti premoženja. Nekatere bolj popularne šole, na primer Harvard, imajo fundacije v vrednosti več bilionov dolarjev, zato v njihovem primeru 5\% predstavlja ogromne količine denarja. \cite{Investopedia}\\

Donatorji lahko omejijo način, kako šole porabijo denar z izjavo o naložbeni politiki (ISP). Na primer, donatorji se lahko odločijo, da bodo uporabili del prihodka iz fundacije za štipendije. Razen teh omejitev lahko univerze preostali del dodeljenega zneska uporabijo kot standardni dohodek. Odločitev o tem, ali ga je potrebno porabiti za najem profesorjev, nadgradnjo oziroma popravilo prostorov ali pa financiranje štipendij, sprejmejo učitelji. \\

Dohodki od naložb fundacije lahko tudi znatno znižajo stroške šolnine za študente. Če na primer finančna sredstva univerze prinesejo skupno 150 milijonov dolarjev in imajo 5-odstotno omejitev porabe, bi to zagotovilo 7,5 milijona dolarjev razpoložljivega dohodka. Če bi univerza prvotno načrtovala 5,5 milijona dolarjev namenjenih sredstev, bi to pomenilo, da bi lahko presežna 2 milijona dolarjev porabila za plačilo drugih dolgov ali stroškov in prihranke lahko prenesla na študente. \cite{Investopedia}\\

Ker pa so univerze odvisne od donosnosti naložb za dodatni dohodek, lahko pride do težav, če naložbe ne prinesejo ustrezne donosnosti. Zato večino fundacij vodijo strokovnjaki.\\


\subsection{Vrste fundacij}
Obstajajo različne vrste fundacij:
\begin{enumerate}
\item \textbf{Terminske fundacije}: te običajno določajo, da se lahko glavnica porabi šele po preteku določenega obdobja ali po določenem dogodku.
\item \textbf{Neomejene fundacije}: to so sredstva, ki jih je mogoče potrošiti, shraniti, naložiti in razdeliti po presoji institucije, ki je prejela darilo.
\item \textit{Quasi-endowment} ali \textbf{navidezna fundacija}: je donacija iz strani posameznika ali institucije, ki je dana z namenom, da ta sklad služi točno določenemu namenu. Glavnica se običajno obdrži, medtem ko se dobiček porabi ali razdeli po specifikacijah donatorja. Te fundacije so največkrat ustanovljene iz strani institucije, ki bo od njih imela korist preko notranjih transakcij ali z uporabo neomejenih sredstev, ki so že bili dodeljeni tej instituciji.
\item \textbf{Omejene fundacije}: njihove glavnice so zadržane za vedno, medtem ko se dobiček od vloženih sredstev porabi po navodilih donatorja.
\end{enumerate}

Razen v kakšnih posebnih okoliščinah, pogojev ni mogoče kršiti. Če je institucija blizu bankrota ali pa ga je že razglasila, vendar ima še vedno premoženje v fundacijah, lahko sodišče izda doktrino o cy-prè-jih(?????), tako da lahko institucija uporabi ta sredstva za boljši finančni položaj, pri tem pa še naprej spoštuje želje donatorja.

\subsection{Razporejenost fundacij po zveznih državah ZDA}

Na sliki 1 so prikazane zvezne države ZDA glede na NACUBO \cite{wiki}, pri čemer svetla barva pomeni, da je v tisti državi majhna skupna vrednost fundacij, medtem ko temnejša barva predstavlja višjo vrednost skupnih fundacij za zasebne univerze. Pri tem moramo vedeti, da so zneski, o katerih govorimo pri fundacijah univerz v ZDA, mišljeni v bilijonih \$. Zelo očitno izstopa zvezna država Kalifornija, ki je obarvana najtemnješe, zato lahko predvidevamo, da se tam skupna vrednost fundacij giblje med 34 in 41 bilijonov \$ - konkretneje dosegajo kar 40.829 bilijona \$. Takoj za Kalifornijo je druga po vrsti zvezna država New York z 29.668 bilijona \$. Nato sledi zvezna država Delaware, Pensilvanija in Illinois. 

\begin{figure}[!h]
\centering
\includegraphics[width = 15 cm]{grafi_zemljevidi/zemljevid_zasebnih.png}
\caption{Fundacije zasebnih univerz v letu 2018, vir: Wikipedia}
\label{Slika 1}
\end{figure}


\begin{figure}[!h]
\centering
\includegraphics[width = 15 cm]{grafi_zemljevidi/zemljevid_javnih.png}
\caption{Fundacije javnih univerz v letu 2018, vir: Wikipedia}
\label{Slika 2}
\end{figure}

Na sliki 2 pa so prikazane skupne vrednosti fundacij javnih univerz po zveznih državah ZDA glede na NACUBO \cite{wiki}. Kot vidimo, zelo jasno iztopa zvezna država Teksas, kjer prejmejo skupno kar 45.790 bilijonov \$ fundacij, precej za Teksasom pa je Kalifornija z 17.666 bilijonov \$ fundacij. Nato sledijo še Pensilvanija, Virginija in Michigan. \\

Na podlagi tega lahko sklepamo, da je v Teksasu največ javnega šolsta, hkrati pa vidimo, da je tudi na zemljevidu fundacij zasebnih šol Teksas nekje na zlati sredini glede na skupno število fundacij v zvezni državi, zato bi lahko rekli, da je v Teksasu šolstvo zelo močno zastopano. Je pa pri tem seveda potrebno upoštevati, da je to tudi precej velika zvezna država. \\

Prav tako v obeh primerih Kalifornija nastopa zelo visoko, kar ponovno poudari dejstvo, da je tam šolstvo zelo pomembno in veliko vlagajo vanj. Če primerjamo oba zemljevida med seboj, opazimo še, da je večina fundacij skoncentriranih na vzhodu ZDA (z izjemo Kalifornije, ki kot že rečeno, v obeh primerih stopa precej visoko).\\

\begin{figure}[!h]
\centering
\includegraphics[width = 15 cm]{grafi_zemljevidi/top10.png}
\caption{Najboljših 10 univerz glede na višino fundacij v letu 2018, vir: Wikipedia}
\label{Slika 3}
\end{figure}

Slika 3 pa nam prikazuje univerze v ZDA z največjimi fundacijami \cite{wiki}. Kot pričakovano, je tukaj na prvem mestu Harvard, nato sledi Yale, Stanfort in Princeton, ki so tudi najbolj poznane. Pri tem vidmo, da je Harvard daleč v ospredju in dosega skoraj 40 bilijonov \$ fundacij v letu 2018 - natančneje 39.428 bilionov \$. Ostale tri omenjene univerze pa se gibljejo precej skupaj, in sicer Yale z 30.314
bilinov \$, Stanford z 27.7 bilijona \$ in Princeton z 26.116 bilijona \$.



\section[Naložbene politike univerzitetnih fundacij]{Naložbene politike univerzitetnih fundacij}

Naložbene politike morajo biti zasnovane tako, da zagotavljajo primerno rast in predvidljivost letnih izplačil. Večna narava sklada zagotavlja tako priložnosti kot izzive. Priložnost je v zmožnosti zajemanja z zapletenih, nelikvidnih ali dolgoročnih naložb, za katere je mogoče pričakovati, da bodo prinesle boljše donose. Zgodovinsko je odgovornost skrbnika sklada zahtevala ohranjanje prvotne vrednosti korpusa, kar je običajno povzročilo velike dodelitve obveznicam, denarnim ekvivalentom in drugim naložbam z nizkim tveganjem. Danes si skrbniki skladov svojo minimalno odgovornost razlagajo kot ohranjanje dejanske vrednosti, prilagojene inflaciji, in letni prenos v poslovni proračun. Ta koncept je znan kot generacijska pravičnost ali \textit{generational equity}, ki nakazuje, da je treba prihodnjim generacijam upravičencev do sredstev zagotoviti vsaj enako raven podpore, kot jo uživajo sedanji upravičenci. Stopnja trajnostne porabe sklada mora biti enaka pričakovani skupni donosnosti obdavčenih sredstev, zmanjšani za predvideno stopnjo inflacije. taka politika zahteva veliko predvidljivost, prvič morajo biti pozorni na napovedovanje inflacije in predvideti morajo skupno donosnost. \\

Ko se dogovorijo glede potrošnje, se morajo dogovoriti glede naložbenih politik. Pogosto se za odločijo, da bodo v naslednjem semestru potrošili 5\% povprečne vrednosti sklada zadnjih treh fiskalnih let. Nekatere institucije se odločijo samo povečati potrošnjo za toliko, kolikor se je povečala inflacija, kar povzroči predvidljiv tok dohodkov na operativni proračun. \cite{investment1}\\

Fundacije velikokrat niso obdavčene, saj so lahko del institucije, ki ni obdavčena, ali pa so same po sebi neobdavčene. To omogoča višje prihodke, saj ne potrebujejo prenesti deleža dohodkov državi. Primerjave se pogosto izvajajo med zasebnimi fundacijami in univerzitetnimi fundaciji, zlasti pri obravnavi določenih politik (na primer zahteve za izplačilo). Za razliko od zasebnih fundacij se za fundacije fakultet ne zahtevajo izplačila. Zasebne fundacije se razlikujejo od javnih dobrodelnih organizacij, ki so oproščene davkov, in sicer po ozkih osnovah nadzora in finančne podpore.\\

V zgodnjih letih prejpnjega desetletja je prišlo do sprememb, kam se vlagajo sredstva fundacij. Med letoma 2002 in 2010 se je delež naložbenih sredstev, vloženih v lastniške deleže, zmanjšal s 50\% na 31\%. Od leta 2010 se je delež premoženjskih sredstev, vloženih v delnice, povečal na 36\%. Odstotek sredstev, vloženih v stalni dohodek, se je v obdobju 2002 do 2017 zmanjšal s 23\% na 8\%. Medtem ko se je delež sredstev, ki se vlagajo v kapital in stalni dohodek, zmanjšal, se je delež sredstev, vloženih v alternativne strategije, povečeval. Empirične raziskave so iskale, zakaj se je dodelitev sredstev univerzitetnih fundacij premaknila v alternativne naložbe. Eden od možnih razlogov je konkurenca. Obstajajo dokazi, da institucije ponavadi povečujejo delež vloženih sredstev v naložbene sklade, da "dohitijo" šole, ki so tekmovalci. Druge raziskave ugotavljajo, da je večja verjetnost, da bodo upravljavci fundacijskih naložb vlagali v alternativna sredstva, če imajo institucije višji in manj variabilen dohodek. Te institucije so morda pripravljene prevzeti večje tveganje, povezano z naložbami v alternativne strategije.\cite{investment2}\\

Glavni dohodek fundacij so pa seveda donacije. Fundacije morajo upoštevati tudi pisne smernice donatorjev v zvezi z dodelitvijo dohodkovnih dajatev za trenutno uporabo. Univerzitetne fundacije imajo tudi dostop do strokovnega znanja, ki ga nudijo investicijski odbori, ki posameznim vlagateljem običajno ni na voljo. Univerze se ponašajo z ogromnimi socialnimi omrežji, ki jim omogočajo večji dostop do številnih ključnih naložbenih priložnosti.
Donacije so tudi oproščene državnih davkov, kar jim daje višje donose. Najbolj uspešne fundacije imajo dostop do alternativnih naložb, ki zahtevajo daljša obdobja gestacije in višje minimalne naložbe, kot si jih lahko privošči večina posameznih vlagateljev. \cite{Investopedia2}

\section[Odzivnost na finančne šoke]{Odzivnist na finančne šoke}

Univerzitetne fundacije kažejo na asimetričen odziv na pozitivne in negativne finančne pretrese. Po pozitivnih šokih ponavadi sledijo njihovim navedenim politikam izplačil (npr. izplačajo 5\% preteklega triletnega povprečja vrednosti dobička fundacije). Medtem ko po negativnih šokih številna sredstva aktivno odstopajo od svojih navedenih pravil o izplačilih, dejansko znižujejo stopnjo izplačil na raven, ki je nižja od običajnih pravil ravnanja. Ne najdemo doslednih dokazov, da univerze spreminjajo izplačila darov, da bi nadomestile šoke v druge vire univerzitetnih prihodkov, torej vedenje fundacij ni v skladu s hipotezami glajenja ali samozavarovanja.\\

V literaturi ni soglasnega stališča o tem, kako določiti ciljne funkcije univerz ali fundacij, ki bi jim pomagale pri njihovi podpori. To pomanjkanje soglasja ni posledica pomanjkanja pozornosti, saj je več vodilnih ekonomistov predlagalo normativne modele obnašanja. \\

Prva teorija o fundacijah trdi, da bi morali skrbniki fundacijskih institucije ravnati, kot da je ustanova nesmrtna, in si prizadevati enako obravnavati vse generacije. Tako njegov model predlaga, da bi se skrbniki morali obnašati, kot da imajo ničelno subjektivno časovno prednost. Trenutna potrošnja ne bi smela imeti koristi od prihodnjih daril v fundacije, prav tako ne bi smele vplivati na spremembe šolnin, nepovratnih sredstev ali drugih prihodkov. Ta model ima dve posledici. Prvič, kratkoročni odziv na izplačila na šoke bi moral biti majhen, ker bi morale fundacije s časom širiti dobičke in izgube. V skrajnem primeru, če se univerzitetna fundacija ukvarja s popolnim ravnanjem v neskončnem življenju, bi se morale fundacije odzvati na trajne šoke s prilagajanjem ravni izplačil glede na vrednost trajnosti šoka. Drugič, fundacije bi se morale simetrično odzvati na pozitivne in negativne pretrese; to pomeni, da bi morali biti njihovi odzivi na pozitivne in negativne pretrese enake.\\

Druga teorija univerzitetnih fundacij je, da univerzam omogočajo samozavarovanje, na primer dodelitev sredstev za varovanje pred šokom prihodkov ali uporabo fundacije kot oblike previdnostnih prihrankov, ki jih je mogoče izkoristiti, ko so drugi prihodki nepričakovano nizki. Na nepopolnih trgih pa bi morale univerze prilagoditi sočasne stopnje izplačil kot odziv na šoke fundacije in druge prihodke. Zato je lahko ključna vloga pri izplačilih dajatev izravnavanje učinkov začasnih pretresov. Če so hitre prilagoditve drage, lahko fundacije zagotovijo dragoceno likvidnost ob stalnih šokih in zmanjšajo skupne stroške prilagoditve univerz. \cite{soki}\\


\section[Uporaba sredstev]{Uporaba sredstev}

\begin{figure}[!h]
\centering
\includegraphics[width = 12 cm]{slike/harvard.png}
\caption{Uporaba sredstev univerze Harvard, vir: Finančno poročilo univerze Harvard}
\label{Slika 4}
\end{figure}


\begin{figure}[!h]
\centering
\includegraphics[width = 14 cm]{slike/cambridge.png}
\caption{Uporaba sredstev univerze Cambridge, vir: Finančno poročilo univerze Cambridge}
\label{Slika 5}
\end{figure}


\newpage
\section[Zaključek]{Zaključek}

Uspešna fundacija lahko pomaga zmanjšati finančno breme univerze z ustvarjanjem stalnega pretoka dohodka. Čeprav fundacije na splošno razkrivajo razčlenitve svojih sredstev, vlagatelji morda ne bodo mogli podvajati uspehov, ki so jih že dosegli.\\

Glavni problem in hkrati prednost univerzitetnih fundacij je izredno dolga ročnost oziroma zmožnost naložitve sredstev v investicije z dolgo ročnostjo, saj je zelo malo primerov, ko bi morala univerza dejansko zapreti fundacijo in se znebiti celotnega premoženja. Tako imajo lahko upravljalci fundacij veliko možnosti kam in kako vlagati vsa sredstva. Ponavadi gre za velike količine premoženja in so lahko posledično investicijski portfoliji fundacij zelo raznoliki.

\newpage
\begin{thebibliography}{99}

\bibitem{Investopedia}
\emph{How do university endowments work?}, v: Investopedia, [ogled 29.~4.~2020], dostopno na \url{https://www.investopedia.com/ask/answers/how-do-university-endowments-work/}.

\bibitem{soki}
Brown, Jeffrey R., et al. \emph{How university endowments respond to financial market shocks: Evidence and implications}, American Economic Review 104.3 (2014): 931-62.

\bibitem{wiki}
\emph{List of colleges and universities in the United States by endowment},  v: Wikipedia: The Free Encyclopedia, [ogled 12.~5.~2020], dostopno na \url{https://en.wikipedia.org/wiki/List_of_colleges_and_universities_in_the_United_States_by_endowment?fbclid=IwAR1FtDeBjPmrv5FimLFX6mvKSw6JftgeRLHL2tfVvxproXHSnaqIfssMkNM}.

\bibitem{investment1}
Spitz, W. T. (2010). \emph{Investment policies for college and university endowments. Roles and Responsibilities of the Chief Financial Officer: New Directions for Higher Education}, Number 107, 99, 51.

\bibitem{investment2}
Sherlock, M. F., Crandall-Hollick, M. L., Gravelle, J., \& Stupak, J. M. (2015). \textit{College and university endowments: Overview and tax policy options. Washington, DC: Congressional Research Service}.

\bibitem{Investopedia2}
\emph{How To Invest Like An Endowment?}, v: Investopedia, [ogled 13.~5.~2020], dostopno na \url{https://www.investopedia.com/articles/financial-theory/09/ivy-league-endowments-money-management.asp}.
\end{thebibliography}



\end{document}
