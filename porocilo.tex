\documentclass[12pt, a4paper]{article}
\usepackage[utf8]{inputenc}
\usepackage[T1]{fontenc}
\usepackage[slovene]{babel}
\usepackage{times}
\renewcommand{\familydefault}{\rmdefault}
\usepackage{amsmath}
\usepackage{eurosym}
\usepackage{hyperref}
\usepackage{graphicx}
\usepackage[top=2.5cm, bottom=2.5cm, left=3cm, right=2.5cm]{geometry}
\usepackage{indentfirst}
\setlength{\parindent}{0.5cm}


\newcommand{\novukaz}[2]{\underline {#1} \textit{#2}}

\newcounter{stevec}

\newenvironment{novookolje}[2]{\stepcounter{stevec} #1 #2 \thestevec}{}

\begin{document}

\begin{titlepage}
\begin{center}

\large
Univerza v Ljubljani\\
\normalsize
Fakulteta za matematiko in fiziko\\

\vspace{3 cm} 

\large
Eva Deželak in Ines Šilc\\

\vspace{0.5cm}
\LARGE
\textbf{Univerzitetne fundacije}

\vspace{0.5 cm}
\normalsize


\vspace{1.5cm}
\normalsize
SEMINARSKA NALOGA PRI PREDMETU FINANČNI TRGI IN INSTITUCIJE

\vspace{3cm}


\vfill

\large Ljubljana, 2020

\end{center}
\end{titlepage}

\newpage

\tableofcontents

\newpage

\listoffigures

\newpage

\section{Uvod}

V seminarski nalogi bova....

Najstarejše fundacije, ki so aktivne še danes, so ustanovili kralj Henry VIII in njegovi sorodniki. Njegova babica, grofica Richmond, je ustanovila "endowed chairs in divinity"????? na Oxfordu in Cambridgu, Henry VIII pa je na obeh ustanovil "professorships in a variety of disciplines"????. Prvo zabeleženo fundacijo pa je ustanovil Marcus Aurelius za višjo šolo filozofije v Atenah leta 176. \cite{Investopedia}

\section{Kaj so univerzitetne fundacije, njihov namen in delovanje}


Fundacije predstavljajo denar ali druga finančna sredstva, ki se donirajo univerzam. Namenjena so investiranju v rast osnovnega kapitala in zagotavljajo dodatni prihodek za prihodnje investicije in izdatke. Običajno fundacijski skladi pri dodeljevanju sredstev sledijo striktnim pravilom z namenom, da ne bi prevzeli preveč tveganja. 

Fundacija je denarna ali premoženjska podpora neprofitni organizaciji, ki porabi dobiček od naložb za določen namen. Večina fundacij je zasnovana tako, da glavni znesek ohrani nedotaknjen, medtem ko donos od naložbe porabi za dobrodelne namene. 

\subsection{Delovanje fundacij}

Večina fundacij ima določene smernice, ki narekujejo, koliko vsakoletnih prihodkov iz naložb lahko porabijo. Na številnih univerzah ta znesek znaša približno 5\% celotne vrednosti premoženja. Nekatere bolj popularne šole, na primer Harvard, imajo fundacije v vrednosti več bilionov dolarjev, zato v njihovem primeru 5\% predstavlja ogromne količine denarja. \cite{Investopedia}

Donatorji lahko omejijo način, kako šole porabijo denar z izjavo o naložbeni politiki (ISP). Na primer, donatorji se lahko odločijo, da bodo uporabili del prihodka iz fundacije za štipendije. Razen teh omejitev lahko univerze preostali del dodeljenega zneska uporabijo kot standardni dohodek. Odločitev o tem, ali ga je potrebno porabiti za najem profesorjev, nadgradnjo oziroma popravilo prostorov ali pa financiranje štipendij, sprejmejo učitelji. 

Dohodki od naložb fundacije lahko tudi znatno znižajo stroške šolnine za študente. Če na primer finančna sredstva univerze prinesejo skupno 150 milijonov dolarjev in imajo 5-odstotno omejitev porabe, bi to zagotovilo 7,5 milijona dolarjev razpoložljivega dohodka. Če bi univerza prvotno načrtovala 5,5 milijona dolarjev namenjenih sredstev, bi to pomenilo, da bi lahko presežna 2 milijona dolarjev porabila za plačilo drugih dolgov ali stroškov in prihranke lahko prenesla na študente. \cite{Investopedia}

Ker pa so univerze odvisne od donosnosti naložb za dodatni dohodek, lahko pride do težav, če naložbe ne prinesejo ustrezne donosnosti. Zato večino fundacij vodijo strokovnjaki.


\subsection{Vrste fundacij}
Obstajajo različne vrste fundacij:
\begin{enumerate}
\item Terminske fundacije: te običajno določajo, da se lahko glavnica porabi šele po preteku določenega obdobja ali po določenem dogodku.
\item Neomejene fundacije: to so sredstva, ki jih je mogoče potrošiti, shraniti, naložiti in razdeliti po presoji institucije, ki je prejela darilo.
\item "quasi-endowment" Navidezna fundacija????: je donacija iz strani posameznika ali institucije, ki je dana z namenom, da ta sklad služi točno določenemu namenu. Glavnica se običajno obdrži, medtem ko se dobiček porabi ali razdeli po specifikacijah donatorja. Te fundacije so največkrat ustanovljene iz strani institucije, ki bo od njih imela korist preko notranjih transakcij ali z uporabo neomejenih sredstev, ki so že bili dodeljeni tej instituciji.
\item Omejene fundacije: njihove glavnice so zadržane za vedno, medtem ko se dobiček od vloženih sredstev porabi po navodilih donatorja.
\end{enumerate}

Razen v kakšnih posebnih okoliščinah, pogojev ni mogoče kršiti. Če je institucija blizu bankrota ali pa ga je že razglasila, vendar ima še vedno premoženje v fundacijah, lahko sodišče izda doktrino o cy-prè-jih(?????), tako da lahko institucija uporabi ta sredstva za boljši finančni položaj, pri tem pa še naprej spoštuje želje donatorja.

\section[Odzivnost na finančne šoke]{Odzivnist na finančne šoke}

Univerzitetne fundacije kažejo na asimetričen odziv na pozitivne in negativne finančne pretrese. Po pozitivnih šokih ponavadi sledijo njihovim navedenim politikam izplačil (npr. izplačajo 5\% preteklega triletnega povprečja vrednosti dobička fundacije). Medtem ko po negativnih šokih številna sredstva aktivno odstopajo od svojih navedenih pravil o izplačilih, dejansko znižujejo stopnjo izplačil na raven, ki je nižja od običajnih pravil ravnanja. Ne najdemo doslednih dokazov, da univerze spreminjajo izplačila darov, da bi nadomestile šoke v druge vire univerzitetnih prihodkov, torej vedenje fundacij ni v skladu s hipotezami glajenja ali samozavarovanja.

V literaturi ni soglasnega stališča o tem, kako določiti ciljne funkcije univerz ali fundacij, ki bi jim pomagale pri njihovi podpori. To pomanjkanje soglasja ni posledica pomanjkanja pozornosti, saj je več vodilnih ekonomistov predlagalo normativne modele obnašanja. 

Prva teorija o obdaritvah izvira iz Tobina (1974) (????), ki trdi, da bi morali skrbniki fundacijskih institucije ravnati, kot da je ustanova nesmrtna, in si prizadevati enako obravnavati vse generacije. Tako njegov model predlaga, da bi se skrbniki morali obnašati, kot da imajo ničelno subjektivno časovno prednost. Trenutna potrošnja ne bi smela imeti koristi od prihodnjih daril v fundacije, prav tako ne bi smele vplivati na spremembe šolnin, nepovratnih sredstev ali drugih prihodkov. Ta model ima dve posledici. Prvič, kratkoročni odziv na izplačila na šoke bi moral biti majhen, ker bi morale fundacije s časom širiti dobičke in izgube. V skrajnem primeru, če se univerzitetna fundacija ukvarja s popolnim ravnanjem v neskončnem življenju, bi se morale fundacije odzvati na trajne šoke s prilagajanjem ravni izplačil glede na vrednost trajnosti šoka. Drugič, fundacije bi se morale simetrično odzvati na pozitivne in negativne pretrese; to pomeni, da bi morali biti njihovi odzivi na pozitivne in negativne pretrese enake.

Druga teorija univerzitetnih fundacij je, da univerzam omogočajo samozavarovanje, na primer dodelitev sredstev za varovanje pred šokom prihodkov ali uporabo fundacije kot oblike previdnostnih prihrankov, ki jih je mogoče izkoristiti, ko so drugi prihodki nepričakovano nizki. Na nepopolnih trgih pa bi morale univerze prilagoditi sočasne stopnje izplačil kot odziv na šoke fundacije in druge prihodke. Zato je lahko ključna vloga pri izplačilih dajatev izravnavanje učinkov začasnih pretresov. Če so hitre prilagoditve drage, lahko fundacije zagotovijo dragoceno likvidnost ob stalnih šokih in zmanjšajo skupne stroške prilagoditve univerz. \cite{soki}

\newpage
\begin{thebibliography}{99}

\bibitem{Investopedia}
\emph{How do university endowments work?}, v: Investopedia, [ogled 29.~4.~2020], dostopno na \url{https://www.investopedia.com/ask/answers/how-do-university-endowments-work/}.

\bibitem{soki}
Brown, Jeffrey R., et al. \emph{How university endowments respond to financial market shocks: Evidence and implications}, American Economic Review 104.3 (2014): 931-62.


\end{thebibliography}



\end{document}
